% -*- root: main.tex -*-
\documentclass[main.tex]{subfiles}
\begin{document}

% ============================================================================

\section{Introduction}
\label{sec:introduction}

Proof reconstruction is a hard labor since it depends on the integration
of two complex systems. On the one hand, we have the automatic theorem provers
(henceforth ATP) and their specification logic. These tools are usually
classified in at least one of the following categories. A SAT solver
(e.g., \verb!zChaff!~\cite{Moskewicz2001} and \verb!MiniSat!~\cite{Een2004})
to prove unsatisfiability of CNF formulas, a QBF solver
(e.g., \verb!GhostQ!~\cite{Klieber2014} and \verb!DepQBF!~\cite{Lonsing2017})
to prove satisfiability and invalidity of quantified Boolean formulas, a SMT
solver (e.g \verb!CVC4!~\cite{Barrett2011}, \verb!veriT!~\cite{bouton2009},
and \verb!Z3!~\cite{DeMoura2008}) to prove unsatisfiability of formulas
from first-order logic with theories, and a prover for validity of
formulas from first-order logic with equality
(e.g., \verb!E!~\cite{Schulz:AICOM-2002}, \verb!leanCoP!~\cite{Otten2008},
\verb!Metis!~\cite{hurd2003first}, \verb!SPASS! ~\cite{Weidenbach2009} and
\verb!Vampire!~\cite{Riazanov1999}), high-order logic (e.g., \verb!Leo-II!
\cite{Benzmuller2008} and \verb!Satallax!~\cite{Brown2012}) or intuitonistic
logic (e.g., \verb!ileanCoP!~\cite{Otten2008}, \verb!JProver!
\cite{Schmitt2001}, and \verb!Gandalf!~\cite{Tammet1997}), among others.
On the other hand, we have the proof checkers, interactive theorem provers
(henceforth ITP) or proof assistants (e.g., \verb!Agda!~\cite{agdateam},
\verb!Coq!~\cite{coqteam}, \verb!Isabelle!~\cite{paulson1994isabelle}, and
\verb!HOL4!~\cite{norrish2007hol}).
The ITP tools provide us the logic framework to check and validate the
reply of the ATPs since they allow us to define the formal language for
the problems like operators, logic variables, axioms, and theorems.

A proof reconstruction tool provides such an integration in one direction,
the bridge between ATPs to ITPs. This is mostly a translation of the reply
delivered by the prover into the formalism of the proof assistant.
Because the formalism of the source (the evidence generated by the ATP) is
not necessarily the same logic in the target, the reconstruction turns out
in a ``reverse engineering'' task. Then, reconstructing a proof involves an
in-depth understanding deep understanding of the algorithms in the ATP and
the specification logic in the ITP.

What we need from the ATP tools is a proof-object in a consistent format
to work with, that is, a full script describing step-by-step with
exhaustive details and without ambiguities, the table of the derivation
to get the actual proof. For problems in classical propositional logic
(henceforth CPL), from a list of at least forty\footnote{ATPs available
from the web service \texttt{SystemOnTPTP} of the TPTP World.} ATPs, just
a few provers were able to deliver proofs (e.g., \verb!CVC4!~
\cite{Barrett2011}, \verb!SPASS!, and
\verb!Waldmeister!~\cite{hillenbrand1997})
and a little bit less replied with a proof in a
file format like \verb!TSTP!~\cite{sutcliffe2004tstp} (e.g., \verb!E!, \verb!Metis!,
\verb!Vampire!, and \verb!Z3!), LFSC~\cite{Stump2008}, or the
SMT-LIB~\cite{Bohme2011} format.

Many approaches have been proposed and some tools have been implemented
for proof reconstruction in the last decades. These programs are relevant
not only because it helps to spread their usage but they also increase
the confidence of their users about their algorithms and their
correctness (see, for example, bugs in ATPs~\cite{Keller2013},
\cite{Bohme2011}, \cite{Fleury2014} and \cite{Kanso2012}).
% We mention some tools in the following.

In this paper, we describe the integration of \verb!Metis! prover with the
proof assistant \verb!Agda! by parsing \verb!TSTP! derivations to generate
\verb!Agda! proof-terms. We structure the paper as follows.
In section \ref{sec:metis-language-and-proofs} we briefly introduce
the \verb!Metis! prover. In section \ref{sec:proof-reconstruction}; we
present our approach to reconstruct proofs deliver by \verb!Metis! in
\verb!Agda!.
%In section \ref{sec:example}, we present a complete example of a proof reconstructed with our tool for a CPL problem.
We list related work in proof reconstruction in section \ref{sec:related-work}
In section~\ref{sec:conclusions}, we discuss some limitations, advantages,
and disadvantages with other similar proof reconstruction tools and
conclusions for ending up with the future work.

\end{document}

\documentclass[../main.tex]{subfiles}
\begin{document}

% ===================================================================

\section{Related Work}
\label{sec:related-work}

Many approaches have been proposed for proof-reconstruction and some tools have
been implemented in the last decades. We first mention some tools in type
theory and later we listed some proof-reconstruction tools for classical logic.

\citeauthor{Kanso2012} in~\cite{Kanso2012,kanso2016light} presents a
proof-reconstruction in \Agda for propositional logic. Its tool support
proof-reconstruction for \name{EProver} and \name{Z3} ATPs following a similar work-flow as we presented in Section~\ref{ssec:workflow}.
Nonetheless, its approach
employs semantics for logic equivalences. We have avoided the use of
propositions meanings towards a future work to support other logics where a
syntactical approach plays an important role~(for an example of such logics, we
refer the reader to \cite{Agudelo-Agudelo2017}).
\citeauthor{foster2011integrating}~\cite{foster2011integrating} describe  the
proof-reconstruction in \Agda for \name{Waldmeister}~\cite{hillenbrand1997}, a
prover for pure equational logic. As far as we know, no other
proof-reconstruction has been carried out neither in \Agda nor with \Metis prover.

Another important proof-assistant in type theory is \name{Coq}~\cite{coqteam}.
We found the \name{SMTCoq}~\cite{armand2011,Ekici2017} tool which provides a
certified checker for proof witness coming from the \SMT solver
\name{veriT}~\cite{bouton2009} and adds a new tactic named verit, that calls
\name{veriT} on any \name{Coq} goal. Also for \name{Coq}, given a fixed but
arbitrary first-order signature, \citeauthor{Bezem2002} in \cite{Bezem2002}
transform a proof produced by the first-order automatic theorem prover
\name{Bliksem}~\cite{deNivelle2003} in a \name{Coq} proof-term.

There are some successful attempts using proof-assistants for classical logic
instead of type theory.
Let us mention some representative of such tools. This description is
mainly based on~\citeauthor{Sicard-Ramirez2016}~\cite{Sicard-Ramirez2016}.

The \name{Isabelle} proof-assistant has the \name{Sledgehammer} tool.
This program provides a full integration between
automatic theorem provers~\cite{blanchette2013extending,Fleury2014,bohme2010} and
\name{Isabelle/HOL}~\cite{nipkow2002isabelle}, the specialization of
\name{Isabelle} for higher-order logic.
A modular proof-reconstruction workflow is presented jointly with
the full integration of \name{Leo-II} and \name{Satallax} provers with
\name{Isabelle/HOL} in \citeauthor{Een2004}~\cite{Een2004}.

\citeauthor{Hurd1999}~\cite{Hurd1999} integrates the first-order resolution
prover \name{Gandalf} with the high-order theorem prover
\name{HOL}~\cite{norrish2017hol}.
Its \verb!GANDALF_TAC! tactic is able to reconstruct \name{Gandalf} proofs
by using a \abbre{LCF} model. For \name{HOL Light}, a version of
\name{HOL} but with a simpler logic core, the \SMT solver \name{CVC4}
was integrated. \citeauthor{kaliszyk2013}~\cite{kaliszyk2013}
reconstruct proofs from different \ATPs with the \name{PRocH} tool by
replaying detailed inference steps from the \ATPs with internal
inference methods implemented in \name{HOL Light}.

\end{document}

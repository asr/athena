% -*- root: main.tex -*-
\documentclass[main.tex]{subfiles}
\begin{document}

% ============================================================================

\section{Related Work}
\label{sec:related-work}

\verb!Sledgehammer! is a tool for \verb!Isabelle! proof assistant
~\cite{paulson1994isabelle} that provides a full integration of automatic
theorem provers including ATPs
(see, for example, \cite{meng2006automation}, \cite{blanchette2013extending}
and \cite{Fleury2014}) and SMT solvers (see, for example,
\cite{hurlin07practical}, \cite{bohme2010},
\cite{blanchette2013extending}, and \cite{Fleury2014}) with
\verb!Isabelle/HOL! \cite{nipkow2002isabelle}, the specialization of
\verb!Isabelle! for higher-order logic. Sultana, Benzm{\"{u}}ller, and
Paulson~\cite{Een2004} integrates \verb!Leo-II! and \verb!Satallax!, two
theorem provers for high-order logic with \verb!Isabelle/HOL! proposing a
modular proof reconstruction work flow.

\verb!SMTCoq!~\cite{armand2011,Ekici2017} is a tool for the \verb!Coq!
proof assistant \cite{coqteam} which provides a certified checker for
proof witness coming from the SMT solver \verb!veriT! \cite{bouton2009}
and adds a new tactic named verit, that calls \verb!veriT! on any
\verb!Coq! goal. In \cite{bezem2002automated}, given a fixed but arbitrary
first-order signature, the authors transform a proof produced by the
first-order automatic theorem prover \verb!Bliksem! \cite{deNivelle2003}
in a \verb!Coq! proof term.

Hurd~\cite{Hurd1999} integrates the first-order resolution prover
\verb!Gandalf! with \verb!HOL!~\cite{norrish2007hol}, a high-order theorem
prover, following a LCF model implementing the tactic \verb!GANDALF_TAC!.
For \verb!HOL Light!, a version of \verb!HOL! but with a simpler logic core,
the SMT solver \verb!CVC4! was integrated and Kaliszyk and
Urban~\cite{kaliszyk2013} reconstruct proofs from different ATPs with their
\verb!PRocH! tool, replaying the detailed inference steps from the ATPs with
internal inference methods implemented in \verb!HOL Light!.

% Taken from: % Färber, M., & Kaliszyk, C. (2015). Metis-based
% Paramodulation Tactic for HOL

% Light. In GCAI 2015. Global Conference on Artificial Intelligence
% Metis-based

%(Vol. 36, pp. 127–136).

% HOL(y)Hammer [KU15] is an automated deduction framework for HOL4 and
% HOL Light.

% Given a conjecture, it attempts to find suitable premises, then calls
% external ATPs such as E[Sch13], Vampire [KV13], or Z3 [dMB08], and
% attempts to reconstruct the proof using the premises used by the ATP.
% To reconstruct proofs, it uses tactics such as MESON, simplification,
% and a
% few other decision procedures, however, these are sometimes not p
% owerful
% enough to reconstruct proofs found by the external ATPs.

\verb!Waldmeister! is an automatic theorem prover for unit equational
logic \cite{hillenbrand1997}.
Foster and Struth~\cite{foster2011integrating} integrate \verb!Waldmeister! into
\verb!Agda!~\cite{agdateam}. This integration requires a proof
reconstruction step but authors' approach is restricted to pure
equational logic --also called identity theory~\cite{humberstone2011}--
that is, first-order logic with equality but no other predicate symbols
and no functions symbols~\cite{appel1959}.

Kanso and Setzer~\cite{kanso2016light} integrate \verb!Z3! in \verb!Agda!
but they also integrate the propositional fragment of the \verb!E! prover in
\cite{Kanso2012}. They cataloged these two integration as
\emph{Oracle and Reflection} and \emph{Oracle and Justification}, respectively.
Their integration with the \verb!E! prover is the most related work with our
apporach found at the moment while we write this document. Besides the use
of the same ITP, crucial differences easily arise like decidability of
propositional logic using semantics that we do not take into account for the
reason we describe later on in section~\ref{sec:conclusions}.

\end{document}
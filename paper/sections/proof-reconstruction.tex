% -*- root: main.tex -*-
\documentclass[../main.tex]{subfiles}
\begin{document}

% ============================================================================

\section{Proof Reconstruction}
\label{sec:proof-reconstruction}

\subsection{The Translation Method}

Our system (see Fig~\ref{fig:proof-reconstruction-overview}) translates a \TSTP
file into an \Agda proof by parsing it with a \Haskell tool.
This tool reconstructs the proof tree, removing redundancies and
cutting unnecessary steps (see for example repetitions in the derivation shown in
Fig~\ref{fig:metis-example}). The output is \Agda code.
Then, we use \Agda as a proof checker to type-check the proof.
If type-checking succeed, then our proof is checked. Else, we have a failure
proof due to three possible factors. First, a soundness bug in the ATP or a bug
in the output printing. Second, a bug in the translation process in our \TSTP parsing
module, the inner algorithms in the analysis of the proof tree, the pretty printer
module, or a bug in \Haskell. Lastly, a bug in our proof checker \Agda.

% ---------------------------------------------------------------------------

\subsection{Reconstruction Work-flow}

\begin{figure}
\begin{tikzpicture}
\node[text width=2cm, align=center](problem) at (0,0)
  {CPL \\ Problem};

\node[right = 1.2cm of problem, text width=2cm, align=center]
  (tptp){\TPTP \\ Problem};

\node[right= 1.2cm of tptp, text width=2cm, align=center]
   (metis) {\Metis \\ Prover};

\node[right= 1.2cm of metis, text width=2cm, align=center, inner sep=10pt]
   (tstp) {\TSTP \\ Derivation};

\node[below= 0.5cm of tstp, text width=2cm, align=center, inner sep=10pt]
  (athena) {\Haskell \\ Traslator};

\node[left = 1.2cm of athena, text width=2cm, align=center]
   (agdafile) {\Agda \\ Proof};

\node[left = 1.2cm of agdafile, text width=2cm, align=center]
   (agda) {Proof Checker};

\node[below = 0.5cm of problem, text width=2cm, align=center]
   (verified) {Checked};

\node[below = 0.5cm of verified, text width=2cm, align=center]
   (failure) {Error};

% node[below] {send to}
\draw[->, thick] (problem) to
  % node[below] {\tiny encoding}
  (tptp);
\draw[->, thick] (tptp) to
  % node[below] {\tiny }
  (metis);
\draw[->, thick] (metis) to
  % node[below] {\tiny replies on}
  (tstp);
\draw[->, thick] (tstp) to
  % node[right] {\tiny parsing}
  (athena);
\draw[->, thick] (athena) to
  % node[below] {\tiny traslation}
  (agdafile);
\draw[->, thick] (agdafile) to
  % node[below] {\tiny type-checking}
  (agda);
\draw[->, thick] (agda) to (verified);
\draw[->, thick, gray] (agda) to (failure);
\end{tikzpicture}
\caption{Proof reconstruction overview.}
\label{fig:proof-reconstruction-overview}
\end{figure}

% ---------------------------------------------------------------------------

\subsection{Proof Cheker}
\label{ssec:agda}\

\Agda is an interactive system for constructing proofs and programs,
based on Martin-L\"{o}f's type theory and extended with records, parametrised
modules, among other features.
One of the main strengths of \Agda is its support for writing proofs,
which we shall call \Agda's proof engine. It consists of: support for
inductively defined types, including inductive families, and function
definitions using pattern matching on such types, normalisation during type-
checking, commands for refining proof terms, coverage checker and termination
checker.
The inductive approach for representing classical propositional logic
is better because we benefit from \Agda's proof engine and its Unicode
support that allows us writing proofs similar as we find in math text books.

% ---------------------------------------------------------------------------

% \subsection{LCF-Style Theorem Proving}
% \label{ssec:lcf-style-theormem-proving}

% A propositional formula is a experssion of indivisible propositional atoms,
% logical connectives $\{\wedge, \vee, \Rightarrow, \Leftrightarrow, \neg\}$
% and logic constants $\{\top, \bot\}$.
% In \Agda, we define the formula
% as an inductive type using the keyword \texttt{data} and including every
% connective as a constructor. We represent propositional formulas with
% \name{Prop} data type using the following definition.

% %an extension of the syntax definition used in \cite{Altenkirch2015}.

% \begin{verbatim}
% data Prop : Set where
%   Var              : Fin n → Prop
%   ⊤                : Prop
%   ⊥                : Prop
%   _∧_ _∨_ _⇒_ _⇔_  : (φ ψ : Prop) → Prop
%   ¬_               : (φ : Prop)   → Prop
% \end{verbatim}

% On the other hand, we represent theorems in\CPLusing an abstract data type
% to implement a natural deduction calculus.

% \begin{verbatim}
% data _⊢_ : (Γ : Ctxt)(φ : Prop) → Set where
% \end{verbatim}

% The sequents  $\Gamma \vdash \phi$ represent theorems,
% where $\Gamma$ is a set of premises and $\phi$ is the
% sequent's conclusion. Strictly speaking, we define the set of premises
% $\Gamma$ with \name{List} data type
% % \footnote{We use the \Agda standard library's implementation of the List data type.}.

% Our\CPLsystem define all inference rules for intuitonistic propositional logic
% (see Fig.~\ref{fig:theorem-constructors}) as constructors of the theorem
% data type, and we postulate the principle of the excluded middle (PEM).

% \begin{verbatim}
% postulate  PEM : ∀ {Γ} {φ} → Γ ⊢ φ ∨ ¬ φ
% \end{verbatim}

% \begin{figure}
% \[%\scalebox{0.9}{
% \begin{bprooftree}
% \AxiomC{}
% \RightLabel{\scriptsize\tt assume $\varphi$}
% \UnaryInfC{$\Gamma , \varphi \vdash \varphi$}
% \end{bprooftree}\quad
% \begin{bprooftree}
% \AxiomC{$\Gamma \vdash \varphi$}
% \RightLabel{\scriptsize\tt weaken $\psi$}
% \UnaryInfC{$\Gamma , \psi \vdash \varphi$}
% \end{bprooftree}\quad
% \begin{bprooftree}
% \AxiomC{}
% \RightLabel{\scriptsize\tt $\top$-intro}
% \UnaryInfC{$\Gamma \vdash \top$}
% \end{bprooftree}\quad
% \begin{bprooftree}
% \AxiomC{$\Gamma \vdash \bot$}
% \RightLabel{\scriptsize\tt $\bot$-elim $\varphi$}
% \UnaryInfC{$\Gamma \vdash \varphi$}
% \end{bprooftree}
% %}
% \]

% \[%\scalebox{0.9}{
% \begin{bprooftree}
% \AxiomC{$\Gamma , \varphi \vdash \bot$}
% \RightLabel{\scriptsize\tt $\neg$-intro}
% \UnaryInfC{$\Gamma \vdash \neg \varphi$}
% \end{bprooftree}\quad
% \begin{bprooftree}
% \AxiomC{$\Gamma \vdash \neg \varphi$}
% \AxiomC{$\Gamma \vdash \varphi$}
% \RightLabel{\scriptsize\tt $\neg$-elim}
% \BinaryInfC{$\Gamma \vdash \bot$}
% \end{bprooftree}\quad
% \begin{bprooftree}
% \AxiomC{$\Gamma \vdash \varphi$}
% \AxiomC{$\Gamma \vdash \psi$}
% \RightLabel{\scriptsize\tt $\wedge$-intro}
% \BinaryInfC{$\Gamma \vdash \varphi \wedge \psi$}
% \end{bprooftree}
% %}
% \]

% \[%\scalebox{0.9}{
% \begin{bprooftree}
% \AxiomC{$\Gamma \vdash \varphi \wedge \psi$}
% \RightLabel{\scriptsize\tt $\wedge$-proj$_1$}
% \UnaryInfC{$\Gamma\vdash \varphi$}
% \end{bprooftree}\quad
% \begin{bprooftree}
% \AxiomC{$\Gamma \vdash \varphi \wedge \psi$}
% \RightLabel{\scriptsize\tt $\wedge$-proj$_2$}
% \UnaryInfC{$\Gamma\vdash \psi$}
% \end{bprooftree}\quad
% \begin{bprooftree}
% \AxiomC{$\Gamma \vdash \varphi$}
% \RightLabel{\scriptsize\tt $\vee$-intro$_1$ $\psi$}
% \UnaryInfC{$\Gamma \vdash \varphi \vee \psi$}
% \end{bprooftree}
% %}
% \]

% \[%\scalebox{0.9}{
% \begin{bprooftree}
% \AxiomC{$\Gamma \vdash \psi$}
% \RightLabel{\scriptsize\tt $\vee$-intro$_2$ $\varphi$}
% \UnaryInfC{$\Gamma \vdash \varphi \vee \psi$}
% \end{bprooftree}\quad
% \begin{bprooftree}
% \AxiomC{$\Gamma ,\varphi \vdash \gamma $}
% \AxiomC{$\Gamma , \psi  \vdash \gamma$}
% \RightLabel{\scriptsize\tt $\vee$-elim}
% \BinaryInfC{$\Gamma , \varphi \vee \psi \vdash \gamma$}
% \end{bprooftree}\quad
% \begin{bprooftree}
% \AxiomC{$\Gamma , \varphi \vdash \psi$}
% \RightLabel{\scriptsize\tt $\Rightarrow$-intro}
% \UnaryInfC{$\Gamma \vdash \varphi \Rightarrow \psi$}
% \end{bprooftree}
% %}
% \]

% \[
% %\scalebox{0.9}{
% \begin{bprooftree}
% \AxiomC{$\Gamma \vdash \varphi \Rightarrow \psi$}
% \AxiomC{$\Gamma \vdash \varphi$}
% \RightLabel{\scriptsize\tt $\Rightarrow$-elim}
% \BinaryInfC{$\Gamma \vdash \psi$}
% \end{bprooftree}\quad
% \begin{bprooftree}
% \AxiomC{$\Gamma , \varphi \vdash \psi$}
% \AxiomC{$\Gamma , \psi \vdash \varphi$}
% \RightLabel{\scriptsize\tt $\Leftrightarrow$-intro}
% \BinaryInfC{$\Gamma \vdash \varphi \Leftrightarrow \psi$}
% \end{bprooftree}
% %}
% \]


% \[
% %\scalebox{0.9}{
% \begin{bprooftree}
% \AxiomC{$\Gamma \vdash \varphi$}
% \AxiomC{$\Gamma \vdash \varphi \Leftrightarrow \psi$}
% \RightLabel{\scriptsize\tt $\Leftrightarrow$-elim$_1$}
% \BinaryInfC{$\Gamma \vdash \psi$}
% \end{bprooftree}\quad
% \begin{bprooftree}
% \AxiomC{$\Gamma \vdash \varphi$}
% \AxiomC{$\Gamma \vdash \varphi \Leftrightarrow \psi$}
% \RightLabel{\scriptsize\tt $\Leftrightarrow$-elim$_2$}
% \BinaryInfC{$\Gamma \vdash \psi$}
% \end{bprooftree}
% %}
% \]

% \caption{Inference rules for\CPLsystem.}
% \label{fig:theorem-constructors}
% \end{figure}

% % ---------------------------------------------------------------------------


% % ---------------------------------------------------------------------------

% \subsection{Emulation of Inference Rules in Agda}
% \label{ssec:emulating-inferences}

% \textit{Normalization.}

% \begin{figure}
% \[ \scalebox{0.9}{
% \begin{bprooftree}
%   \AxiomC{$\Gamma \vdash \varphi$}

%   \UnaryInfC{$\Gamma \vdash \texttt{cnf}~\varphi$}
% \end{bprooftree}
% \qquad
% \begin{bprooftree}
%   \AxiomC{$\Gamma \vdash \varphi$}
%   \UnaryInfC{$\Gamma \vdash \texttt{nnf}~\varphi$}
% \end{bprooftree}
% \qquad
% \begin{bprooftree}
%   \AxiomC{$\Gamma \vdash \varphi$}
%   \UnaryInfC{$\Gamma \vdash \texttt{dnf}~\varphi$}
% \end{bprooftree}
% }\]
% \end{figure}

% \begin{figure}
% \[%\scalebox{0.9}{
% \begin{bprooftree}
%   \AxiomC{$\Gamma \vdash \varphi \wedge \bot$}
%   \UnaryInfC{$\Gamma \vdash \bot$}
% \end{bprooftree}
% \begin{bprooftree}
%   \AxiomC{$\Gamma \vdash \varphi \wedge \top$}
%   \UnaryInfC{$\Gamma \vdash \varphi$}
% \end{bprooftree}
% \begin{bprooftree}
%   \AxiomC{$\Gamma \vdash \varphi \wedge \neg \varphi$}
%   \UnaryInfC{$\Gamma \vdash \bot$}
% \end{bprooftree}
% %}
% \]

% \[%\scalebox{0.9}{
% \begin{bprooftree}
%   \AxiomC{$\Gamma \vdash \varphi \vee \bot$}
%   \UnaryInfC{$\Gamma \vdash \varphi$}
% \end{bprooftree}
% \qquad
% \begin{bprooftree}
%   \AxiomC{$\Gamma \vdash \varphi \vee \top$}
%   \UnaryInfC{$\Gamma \vdash \top$}
% \end{bprooftree}
% \qquad
% \begin{bprooftree}
%   \AxiomC{$\Gamma \vdash \varphi \vee \neg \varphi$}
%   \UnaryInfC{$\Gamma \vdash \top$}
% \end{bprooftree}
% %}
% \]
% \end{figure}

% \textit{Resolution.}

% \begin{figure}
% \[%\scalebox{0.9}{
% \begin{bprooftree}
% \AxiomC{$\delta_{\phi} := \Gamma \vdash \varphi$}
% \AxiomC{$\delta_{\psi} := \Gamma \vdash \psi$}
% \RightLabel{\texttt{resolve} $p$}
% \BinaryInfC{$\Gamma \vdash$ \texttt{resolve} $\phi\ \psi\ p$}
% \end{bprooftree}
% %}
% \]
% % \caption{Resolution in \Metis.}
% \label{fig:resolution-rule}
% \end{figure}

% \textit{Splitting a conjunct.}
% In the following theorem, we see the definition for the \name{conjunct} rule.
% It extracts the $\varphi_i$ conjunct from the conjunction given by $\varphi$.

% \[\scalebox{0.9}{
% \begin{bprooftree}
% \AxiomC{$\Gamma \vdash \varphi := \varphi_1 \wedge \cdots\wedge\varphi_i
% \wedge\cdots\wedge \varphi_n$}
% \AxiomC{$\varphi_i \equiv \psi$}
% \RightLabel{\footnotesize\tt conjunct $\psi$}
% \BinaryInfC{$\Gamma\vdash\varphi_i$}
% \end{bprooftree}
% }
% \]


% \textit{Splitting a goal}


% \begin{figure}
% \[\scalebox{0.9}{
% \begin{bprooftree}
% \AxiomC{$\Gamma \vdash \varphi_1 \wedge \varphi_2$}
% \UnaryInfC{$\Gamma \vdash \texttt{unshunt}~(\texttt{split}~\varphi_1) \wedge \texttt{unshunt}~(\varphi_1 \Rightarrow \texttt{split}~\varphi_2)$}
% \end{bprooftree}
% }\]

% \[\scalebox{0.9}{
% \begin{bprooftree}
% \AxiomC{$\Gamma \vdash \varphi_1 \vee \varphi_2$}
% \UnaryInfC{$\Gamma \vdash \texttt{unshunt}~(\neg \varphi_1 \Rightarrow \texttt{split}~\varphi_2)$}
% \end{bprooftree}
% }\]

% \[\scalebox{0.9}{
% \begin{bprooftree}
% \AxiomC{$\Gamma \vdash \varphi_1 \Rightarrow \varphi_2$}
% \UnaryInfC{$\Gamma \vdash \texttt{unshunt}~(\varphi_1 \Rightarrow \texttt{split}~\varphi_2)$}
% \end{bprooftree}\qquad
% \begin{bprooftree}
% \AxiomC{$\Gamma \vdash \neg \neg \varphi$}
% \UnaryInfC{$\Gamma \vdash \texttt{unshunt}~(\texttt{split}~\varphi)$}
% \end{bprooftree}
% }\]

% \[\scalebox{0.9}{
% \begin{bprooftree}
% \AxiomC{$ \Gamma \vdash \varphi_1 \Leftrightarrow \varphi_2$}
% \UnaryInfC{$\Gamma\vdash\texttt{unshunt}~(\varphi_1 \Rightarrow \texttt{split}~\varphi_2) \wedge \texttt{unshunt}~(\varphi_2 \Rightarrow \texttt{split}~\varphi_1)$}
% \end{bprooftree}
% }\]

% \[\scalebox{0.9}{
% \begin{bprooftree}
% \AxiomC{$\Gamma \vdash \neg \top$}
% \UnaryInfC{$\Gamma \vdash \bot$}
% \end{bprooftree}\qquad
% \begin{bprooftree}
% \AxiomC{$\Gamma \vdash \neg \bot$}
% \UnaryInfC{$\Gamma \vdash \top$}
% \end{bprooftree}\qquad
% \begin{bprooftree}
% \AxiomC{$\Gamma \vdash \neg (\varphi_1 \wedge \varphi_2)$}
% \UnaryInfC{$\Gamma \vdash \texttt{unshunt} (\varphi_1 \Rightarrow \texttt{split}~ \neg \varphi_2)$}
% \end{bprooftree}
% }\]

% \[\scalebox{0.9}{
% \begin{bprooftree}
% \AxiomC{$\Gamma \vdash \neg (\varphi_1 \vee \varphi_2)$}
% \UnaryInfC{$\Gamma \vdash \texttt{unshunt}~(\texttt{split}~\neg \varphi_1) \wedge  \texttt{unshunt}~(\neg \varphi_1 \Rightarrow \texttt{split}~\neg \varphi_2)$}
% \end{bprooftree}
% }\]

% \[\scalebox{0.9}{
% \begin{bprooftree}
% \AxiomC{$\Gamma \vdash \neg (\varphi_1 \Rightarrow \varphi_2)$}
% \UnaryInfC{$\Gamma \vdash \texttt{unshunt}~(\texttt{split}~\varphi_1) \wedge \texttt{unshunt}~(\varphi_1 \Rightarrow \texttt{split}~\neg \varphi_2)$}
% \end{bprooftree}\qquad
% }\]

% \[\scalebox{0.9}{
% \begin{bprooftree}
% \AxiomC{$\Gamma \vdash \neg (\varphi_1 \Leftrightarrow \varphi_2)$}
% \UnaryInfC{$\Gamma \vdash \texttt{unshunt}~(\varphi_1 \Rightarrow \texttt{split}~\neg \varphi_2) \wedge \texttt{unshunt}~(\neg \varphi_2 \Rightarrow \texttt{split}~\varphi_1)$}
% \end{bprooftree}
% }\]
% \caption{Split a goal in an equivalent conjunction of subgoals.}
% \label{split}
% \end{figure}

% In the theorems above appear two recursive functions \texttt{unshunt} and \texttt{split}. They have the following definitions.

% \begin{verbatim}
% unshunt : Prop → Prop
% unshunt (φ₁ ⇒ (φ₂ ⇒ φ₃)) = unshunt ((φ₁ ∧ φ₂) ⇒ φ₃)
% unshunt (φ₁ ⇒ (φ₂ ∧ φ₃)) = unshunt (φ₁ ⇒ φ₂)) ∧ (unshunt(φ₁ ⇒ φ₃)
% unshunt φ                = φ
% \end{verbatim}

% \begin{verbatim}
% split : Prop → Prop
% split (φ₁ ∧ φ₂)     = {!!}
% split (φ₁ ∨ φ₂)     = {!!}
% split (φ₁ ⇒ φ₂)     = {!!}
% split (φ₁ ⇔ φ₂)     = {!!}
% split (¬ ⊤)         = {!!}
% split (¬ ⊥)         = {!!}
% split (¬ (φ₁ ∧ φ₂)) = {!!}
% split (¬ (φ₁ ∨ φ₂)) = {!!}
% split (¬ (φ₁ ⇒ φ₂)) = {!!}
% split (¬ (φ₁ ⇔ φ₂)) = {!!}
% split (¬ (¬ φ))     = {!!}
% split φ₁            = {!!}
% \end{verbatim}

% ...

% \textit{Simplification.}

% \begin{figure}
% \[%\scalebox{0.9}{
% \begin{bprooftree}
%   \AxiomC{$\Gamma \vdash \varphi \Leftrightarrow \psi$}
%   \AxiomC{$\Gamma \vdash \psi$}
%   \RightLabel{\footnotesize\tt $\Leftrightarrow$-elim}
%   \BinaryInfC{$\Gamma \vdash \varphi$}
% \end{bprooftree}
% \begin{bprooftree}
%   \AxiomC{$\Gamma \vdash \varphi \Rightarrow \psi$}
%   \AxiomC{$\Gamma \vdash \varphi$}
%   \RightLabel{\footnotesize\tt $\Rightarrow$-elim}
%   \BinaryInfC{$\Gamma \vdash \psi$}
% \end{bprooftree}
% \]
% \[
% \begin{bprooftree}
%   \AxiomC{$\Gamma \vdash \varphi \vee \neg \psi$}
%   \AxiomC{$\Gamma \vdash \neg \varphi$}
%   \BinaryInfC{$\Gamma \vdash \psi$}
% \end{bprooftree}
% %}
% \]

% \caption{Additional simplification theorems.}
% \label{fig:simplify-rule}
% \end{figure}

\end{document}
\documentclass[../../main.tex]{subfiles}
\begin{document}


\subsubsection{Clausify.}
\label{sssec:clausification}

The \clausify rule is an alternative rule to transform a formula into
its clausal normal form but without performing simplifications with
tautologies or definitions. Recall, this kind of conversion was
addressed by the \canonicalize rule. It is important to notice that
this kind of conversions between one formula to its clausal normal
form are not unique, and \Metis has customized approaches to perform
that transformations. Therefore, we perform a reordering of the
conjunctive normal form given by the $\fcnf$ function defined in
Lemma~\ref{lem:cnf} with the $\freorder_{∧∨}$ function from
Lemma~\ref{lem:reorder-and-or} to the input formula of the rule.

\begin{myexamplenum}
In the following \TSTP derivation by \Metis, we see how
\clausify transforms the \texttt{n₀} formula to get \texttt{n₁} formula.

\begin{verbatim}
  fof(n₀, ¬ p ∨ (q ∧ r) ...
  fof(n₁, (¬ p ∨ q) ∧ (¬ p ∨ r), inf(clausify, n₀)).
\end{verbatim}

\end{myexamplenum}

\begin{mainth}
\label{thm:clausify}
   Let $ψ : \Target$. If $Γ ⊢ φ$ then $Γ ⊢ \fclausify~φ~ψ$, where
  \begin{equation*}
  \begin{aligned}
  &\hspace{.495mm}\fclausify : \Source → \Target → \Prop\\
  &\begin{array}{llll}
  \fclausify~φ~ψ &=
         \begin{cases}
        ψ, &\text{ if }φ≡ψ;\\
        \freorder_{∧∨}~(\fcnf~φ)~ψ, &\text{ otherwise.}
      \end{cases}
  \end{array}
  \end{aligned}
  \end{equation*}
\end{mainth}

\begin{proof}
If $φ ≡ ψ$, $Γ ⊢ \fclausify~φ~ψ$ normalizes to $Γ ⊢ ψ$. The conclusion follows by applying the $\fsubst$ lemma. Otherwise, we use Lemma~\ref{lem:reorder-and-or} and Lemma~\ref{lem:cnf}.
\end{proof}

\end{document}

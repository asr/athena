\documentclass[../../main.tex]{subfiles}
\begin{document}

\subsubsection{Splitting a conjunct.}
\label{sssec:splitting-a-conjunct}

The \conjunct rule extracts from a
conjunction one of its conjuncts. This rule is a generalization of the
projection rules for the conjunction connective as the
following \TSTP excerpt shows.

\begin{verbatim}
fof(normalize_1, plain, p & q & (r | ~ p),
  inference(canonicalize, [], [x])).
fof(normalize_2, plain, q,
  inference(conjunct, [], [normalize_1])).
fof(normalize_2, plain, (r | ~ p),
  inference(conjunct, [], [normalize_1])).
\end{verbatim}

\begin{definition}[conjunct]
  \label{def:conjunct}
  \begin{align*}
    \begin{split}
    &\fconjunct : \Prop \to \Prop \to \Prop\\
    &\fconjunct(φ, ψ) =
      \begin{cases}
          ψ, &\text{ if }φ ≡ ψ\\
          ψ, &\text{ if }φ ≡ φ₁ ∧ φ₂\text{ , }ψ ≡ \fconjunct(φ₁, ψ)\\
          ψ, &\text{ if }φ ≡ φ₁ ∧ φ₂\text{ , }ψ ≡ \fconjunct(φ₂, ψ)\\
          φ, &\text{ otherwise.}
        \end{cases}
    \end{split}
  \end{align*}
\end{definition}

\begin{theorem}[thm-conjunct]
  \label{thm:thm-conjunct}
  If $Γ ⊢ φ$ and $ψ  : \Prop$ then $Γ ⊢ \fconjunct(φ, ψ)$.
\end{theorem}
\begin{proof}\hspace{2cm}
\begin{itemize}
  \item For the case $φ ≡ ψ$, $Γ ⊢ \fconjunct(φ, ψ)$ normalizes to $Γ ⊢ ψ$.
Then, we get the desire conclusion by applying the $\fsubst$ theorem.
  \item If the proposition $φ$ is a conjunction, and we can get $ψ ≡ \fconjunct(φ_{i}, ψ)$ for some $i = 1,\ 2$, then,

\begin{equation*}
  \begin{bprooftree}
  \AxiomC{$Γ ⊢ φ₁ ∧ φ₂$}
  \RightLabel{∧-proj$_{i}$}
  \UnaryInfC{$Γ ⊢ φ_{i}$}
  \UnaryInfC{$Γ ⊢ \fconjunct(φ_{i}, ψ)$}
  \AxiomC{$ψ ≡ \fconjunct(φ_{i}, ψ)$}
  \RightLabel{subst.}
  \BinaryInfC{$Γ ⊢ ψ$}
  \end{bprooftree}
\end{equation*}
\item The last case follows trivially by using the same hypothesis.
\end{itemize}
\end{proof}
\end{document}

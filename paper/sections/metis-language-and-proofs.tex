% % -*- root: main.tex -*-
\documentclass[../main.tex]{subfiles}
\begin{document}

% ===========================================================================

\section{Metis: Language and Proofs}
\label{sec:metis-language-and-proofs}

\verb!Metis! is an automatic theorem prover written in Standard ML for
first-order logic with equality developed by John Hurd~\cite{hurd2003first}.
This prover is suitable for proof reconstruction task since it provides
well-documented proof-objects with enough information to justify his deduction
steps and it also has been used by other proof-reconstruction tools.

\verb!Isabelle/HOL! integrated \verb!Metis! as a macro-step reconstruction
tool for explaining proof steps (usually CNF goals) replied from other ATPs like
\verb!CVC4!, \verb!Vampire!, or \verb!Z3!.
In \cite{Farber2015}, \verb!Metis! was used to provide a tactic for
\verb!HOL Light!; this tactic challenges others like \verb!MESON! tactic or the
\verb!leanCoP! tactic that reconstructs proofs delivered by
\verb!leanCoP!~\cite{Farber2016}.

In this paper, we ported a subset of \verb!Metis!' inference rules to \verb!Agda!,
the propositional fragmented, to allow us to justify step-by-step the proofs
delivered in \verb!TSTP! format generated by \verb!Metis!.

% ---------------------------------------------------------------------------

\subsection{Input and Output Language}
\label{ssec:input-and-output-language}

\textit{Input.}~The \verb!TPTP! language  --which includes the first-order
form (denoted by \verb!fof!) and clause normal form (denoted by \verb!cnf!)
formats~\cite{sutcliffe2009} -- is de
facto input standard language to encode problems for many ATPs.
The \verb!TPTP! syntax describes a well-defined grammar to handle annotated
formulas with the following form

\begin{verbatim}
language(name, role, formula).
\end{verbatim}

The \verb!language! can be \verb!fof! or \verb!cnf!. The \verb!name!
serves to identify the formula within the problem. Each formula assumes one
\verb!role!, this could be an \verb!axiom!, \verb!conjecture!,
\verb!definition!, or a \verb!hypothesis!. The formulas include Boolean
connectives (\verb!&!, \verb!|!, \verb!=>!, \verb!<=>!, $\tt\sim$) and logic
constants \verb!$true! and \verb!$false!.

For instance, we express the problem
$p\, \vdash \neg (p \wedge \neg p) \vee (q \vee \neg q)$
in \verb!TPTP! syntax as follows.

\begin{verbatim}
fof(h, axiom, p).
fof(goal, conjecture, ~ ((p & ~ p) | (q & ~ q))).
\end{verbatim}

\textit{Output.}~\verb!TSTP! language is de facto output standard language
for derivations of ATPs~\cite{sutcliffe2004tstp}.
A \verb!TSTP! derivation is a directed acyclic graph, a proof tree, where each leaf
is a formula from the \verb!TPTP! input. A node is a formula inferred from the parent
formulas. The root is the final derived formula. Such a derivation is a list of
annotated formulas with the form

\begin{verbatim}
language(name, role, formula, source [,useful info]).
\end{verbatim}

The \verb!source! field is an inference record with the following pattern
\begin{verbatim}
inference(rule, useful info, parents).
\end{verbatim}

The \verb!rule! stands for the inference name; the other fields are
supporting arguments or useful information to apply the reasoning step, and
the list of parents nodes.
For instance, in Fig.~\ref{fig:metis-proof-tstp}, \verb!strip! is the name of one
of the inference with no arguments and only one parent node, the \verb!goal!.

\begin{figure}
\begin{verbatim}
fof(a, axiom, p) .
fof(goal, conjecture, p) .
fof(subgoal_0, plain, p, inference (strip, [], [goal])) .
fof(negate_0_0, plain, ~ p, inference (negate, [], [subgoal_0])) .
fof(normalize_0_0, plain, ∼ p,
  inference (canonicalize, [], [negate_0_0])) .
fof(normalize_0_1, plain, p,
  inference (canonicalize, [], [a])) .
fof(normalize_0_2, plain, $false,
  inference (simplify, [], [normalize_0_0, normalize_0_1]))
cnf(refute_0_0, plain, $false,
  inference (canonicalize, [], [normalize_0_2])) .
\end{verbatim}
\caption{A simple \texttt{Metis}' \texttt{TSTP} derivation for the problem $p\vdash p$.}
\label{fig:metis-proof-tstp}
\end{figure}

% ---------------------------------------------------------------------------

\subsection{Proofs}
\label{ssec:metis-proofs}

A proof-object delivered in a \verb!Metis!' proof encodes a natural
deduction proof. The deduction system uses six inference rules
\cite{hurd2003first} (see Fig.~\ref{fig:metis-inferences}) to attempt to
prove conjectures by refutation (see. e.g., the last conclusion
in the TSTP derivation Fig.~\ref{fig:metis-example} is \emph{falsium}).

\begin{figure}
\[
% \scalebox{0.9}{
\begin{bprooftree}
  \AxiomC{}
  \RightLabel{axiom}
  \UnaryInfC{$C$}
\end{bprooftree}
\qquad
\begin{bprooftree}
  \AxiomC{}
  \RightLabel{assume $L$}
  \UnaryInfC{$L \vee \neg L$}
\end{bprooftree}
\qquad
\begin{bprooftree}
  \AxiomC{}
  \RightLabel{refl $t$}
  \UnaryInfC{$t = t$}
\end{bprooftree}
\qquad
\begin{bprooftree}
  \AxiomC{$C$}
  \RightLabel{subst $\sigma$}
  \UnaryInfC{$\sigma\,C$}
\end{bprooftree}
\qquad
% }
\]
\[
% \scalebox{0.9}{
\begin{bprooftree}
  \AxiomC{}
  \RightLabel{equality $L$ $p$ $t$}
  \UnaryInfC{$\neg (L[p] = t) \vee \neg L \vee L[ p \mapsto t]$}
\end{bprooftree}
\qquad
\begin{bprooftree}
  \AxiomC{$L \vee C$}
  \AxiomC{$\neg L \vee D$}
  \RightLabel{resolve $L$}
  \BinaryInfC{$C \vee D$}
\end{bprooftree}
% }
\]
\caption{Inference rules of the \texttt{Metis} prover.}
\label{fig:metis-inferences}
\end{figure}

These proofs are directed acyclic graphs, trees of refutations. Each node stands
for an application of an inference rule and the
leaves in the tree represent formulas in the given problem. Each node is
labeled with a name of the inference rule
(e.g., \verb!canonicalize! in Fig.~\ref{fig:metis-example}). Each
edge links a premise with one conclusion. The proof graphs have at their
root the conclusion $\bot$ since \verb!Metis! delivers proof by refutation.

\begin{figure}
\centering
\begin{bprooftree}\tt
  \AxiomC{}
  \RightLabel{\scriptsize negate}
  \UnaryInfC{$\neg p$}
  \RightLabel{\scriptsize strip}
  \UnaryInfC{$\neg p$}
  \AxiomC{}
  \RightLabel{\scriptsize axiom}
  \UnaryInfC{$p$}
  \RightLabel{\scriptsize canonicalize}
  \UnaryInfC{$p$}
  \RightLabel{\scriptsize simplify}
  \BinaryInfC{$\bot$}
  \RightLabel{\scriptsize canonicalize}
  \UnaryInfC{$\bot$}
\end{bprooftree}
% \caption{The \verb!Metis!' refutation tree for $p \vdash p$ from the
% derivation in Fig.~\ref{fig:metis-proof-tstp}}
\label{fig:metis-example}
\end{figure}

% ---------------------------------------------------------------------------

\subsection{Proof Rules}
\label{ssec:proof-rules}

Despite the six rules in the \verb!Metis!' logic
kernel (see~Fig. \ref{fig:metis-inferences}), we observed in \verb!TSTP!
derivations for CPL problems other inference rules.
They are \verb!canonicalize!, \verb!conjunct!, \verb!negate!, \verb!simplify!,
\verb!strip! and \verb!resolve!. We briefly review each of these rules
following no order except maybe by their frequency in the \verb!TSTP! derivations.

% ...........................................................................

\textit{Splitting}.
To prove a goal, \verb!Metis! splits the goal into disjoint cases. This
process produces a list of new subgoals, the conjunction of these subgoals
implies the goal as we show later in subsection~\ref{ssec:emulating-inferences}.
Then, a proof of the goal becomes in smaller proofs, one
refutation for each subgoal. These subgoals are introduced in the \verb!TSTP!
derivation with the \verb!strip! inference rule.

\begin{verbatim}
fof(goal, conjecture, p & r & q).
fof(subgoal_0, plain, p, inference(strip, [], [goal])).
fof(subgoal_1, plain, p => r, inference(strip, [], [goal])).
fof(subgoal_2, plain, (p & r) => q, inference(strip, [], [goal])).
\end{verbatim}

% ...........................................................................

\textit{Clausification.} The \verb!clausify! rule transforms a
propositional formula in clausal normal form or CNF,
a conjunction of clauses. Where a clause is the disjunction of zero or
more literals and a literal is an atom (positive literal) or a negation of an
atom (negative literal). This conversion is not unique, and \verb!Metis! performs
this transformation using its criteria.

% ...........................................................................

\textit{Normalization.}
\verb!Metis! uses the \verb!canonicalize! rule to normalize a
propositional formula that comes from an axiom or a definition.
It often converts each expression in one of its normal form,
the conjunctive normal form, the negative normal form or the disjunctive normal form.
Posterior, this rule simplifies the formula with some definitions, applying
recursively on the structure's formula conjunctions and disjunctions simplifications
to remove tautologies
(see Fig.~\ref{fig:conjunctive-disjunctive-simplification} with a list
of these theorems. We had assumed the commutative property for conjunction and
disjunction connectives).

\begin{figure}
\[%\scalebox{0.9}{
  \begin{bprooftree}
    \AxiomC{$\varphi \wedge \bot$}
    \UnaryInfC{$\bot$}
  \end{bprooftree}
  \qquad
  \begin{bprooftree}
    \AxiomC{$\varphi \wedge \top$}
    \UnaryInfC{$\varphi$}
  \end{bprooftree}
  \qquad
  \begin{bprooftree}
    \AxiomC{$\varphi \wedge \neg \varphi$}
    \UnaryInfC{$\bot$}
  \end{bprooftree}
  \qquad
  \begin{bprooftree}
    \AxiomC{$\varphi \wedge \varphi$}
    \UnaryInfC{$\varphi$}
  \end{bprooftree}
%}
\]

\[%\scalebox{0.9}{
  \begin{bprooftree}
    \AxiomC{$\varphi \vee \bot$}
    \UnaryInfC{$\varphi$}
  \end{bprooftree}
  \qquad
  \begin{bprooftree}
    \AxiomC{$\varphi \vee \top$}
    \UnaryInfC{$\top$}
  \end{bprooftree}
  \qquad
  \begin{bprooftree}
    \AxiomC{$\varphi \vee \neg \varphi$}
    \UnaryInfC{$\top$}
  \end{bprooftree}
  \qquad
  \begin{bprooftree}
    \AxiomC{$\varphi \vee \varphi$}
    \UnaryInfC{$\varphi$}
  \end{bprooftree}
%}
\]
% \caption{Some rules of \verb!canonicalize! inference.}
\label{fig:conjunctive-disjunctive-simplification}
\end{figure}

% ...........................................................................

\textit{Resolution.} The \verb!resolve! rule is the resolution
theorem showed in Fig.~\ref{fig:metis-inferences}. Its applications
needs the positive literal for resolution and two
derivations to deduce the \emph{resolvent}.
The positive literal $p$ must occur in the formula from the first derivation,
and the negative literal must occur in the formula from the second derivation
(this follows the same pattern in \verb!Metis!' logic core in
Fig.~\ref{fig:metis-inferences} for the resolve inference).

\begin{verbatim}
cnf(refute_0_0, plain, p | q,
  inference(canonicalize, [], [normalize_0_0])).
cnf(refute_0_1, plain, ~ p,
  inference(canonicalize, [], [normalize_0_4])).
cnf(refute_0_2, plain, q,
  inference(resolve, [$cnf(p)], [refute_0_0, refute_0_1])).
\end{verbatim}

% ...........................................................................

\textit{Splitting a conjunct.}
The \verb!conjunct! rule extracts from a conjunction one of its conjuncts; it is
a generalization of the projection rules for the conjunction connective as
we can see in the following \verb!TSTP! derivation.

\begin{verbatim}
fof(normalize_1, plain, p & q & r,
  inference(canonicalize, [], [x])).
fof(normalize_2, plain, q,
  inference(conjunct, [], [normalize_1])).
\end{verbatim}

\textit{Negate.}
Each subgoal proof is a refutation, thereby each proof assumes
the negation of its subgoal. The \verb!negate! rule
introduces the negation of a subgoal that results
after applies the \verb!strip! inference to the goal.

\begin{verbatim}
fof(subgoal_0, plain, p,
  inference(strip, [], [goal])).
fof(negate_0_0, plain, ~ p,
  inference(negate, [], [subgoal_0])).
\end{verbatim}

% ...........................................................................

\textit{Simplification.} %abbreviate formulas
The \verb!simplify! rule is a $n$-ary inference that performs simplification
of definitions. This rule transverses a list of formulas by
applying different theorems (e.g., \emph{modus pones}, \emph{modus tollens},
or \emph{disjunctive syllogism}) including the list of theorems used by
\verb!canonicalize! in Fig.~\ref{fig:conjunctive-disjunctive-simplification}
and the resolution theorem of \verb!resolve! rule.
Moreover, many things happen inside this \verb!Metis!' procedure, and
it is somewhat complicated to grasp it completely.

\end{document}